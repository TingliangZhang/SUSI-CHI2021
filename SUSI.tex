%%
%% This is file `sample-manuscript.tex',
%% generated with the docstrip utility.
%%
%% The original source files were:
%%
%% samples.dtx  (with options: `manuscript')
%% 
%% IMPORTANT NOTICE:
%% 
%% For the copyright see the source file.
%% 
%% Any modified versions of this file must be renamed
%% with new filenames distinct from sample-manuscript.tex.
%% 
%% For distribution of the original source see the terms
%% for copying and modification in the file samples.dtx.
%% 
%% This generated file may be distributed as long as the
%% original source files, as listed above, are part of the
%% same distribution. (The sources need not necessarily be
%% in the same archive or directory.)
%%
%% The first command in your LaTeX source must be the \documentclass command.
\documentclass[manuscript,screen]{acmart}

%%
%% \BibTeX command to typeset BibTeX logo in the docs
\AtBeginDocument{%
  \providecommand\BibTeX{{%
    \normalfont B\kern-0.5em{\scshape i\kern-0.25em b}\kern-0.8em\TeX}}}

%% Rights management information.  This information is sent to you
%% when you complete the rights form.  These commands have SAMPLE
%% values in them; it is your responsibility as an author to replace
%% the commands and values with those provided to you when you
%% complete the rights form.
% \setcopyright{acmcopyright}
% \copyrightyear{2020}
% \acmYear{2020}
% \acmDOI{10.1145/1122445.1122456}


%% These commands are for a PROCEEDINGS abstract or paper.


% \acmConference[Woodstock '18]{Woodstock '18: ACM Symposium on Neural
%   Gaze Detection}{June 03--05, 2018}{Woodstock, NY}
% \acmBooktitle{Woodstock '18: ACM Symposium on Neural Gaze Detection,
%   June 03--05, 2018, Woodstock, NY}
% \acmPrice{15.00}
% \acmISBN{978-1-4503-XXXX-X/18/06}


%%
%% Submission ID.
%% Use this when submitting an article to a sponsored event. You'll
%% receive a unique submission ID from the organizers
%% of the event, and this ID should be used as the parameter to this command.
%%\acmSubmissionID{123-A56-BU3}

%%
%% The majority of ACM publications use numbered citations and
%% references.  The command \citestyle{authoryear} switches to the
%% "author year" style.
%%
%% If you are preparing content for an event
%% sponsored by ACM SIGGRAPH, you must use the "author year" style of
%% citations and references.
%% Uncommenting
%% the next command will enable that style.
%%\citestyle{acmauthoryear}

%%
%% end of the preamble, start of the body of the document source.
\begin{document}

%%
%% The "title" command has an optional parameter,
%% allowing the author to define a "short title" to be used in page headers.
\title{SUSI GENE: a portable robot as venting, recording  and sharing tool for improving mental health condition}

%%
%% The "author" command and its associated commands are used to define
%% the authors and their affiliations.
%% Of note is the shared affiliation of the first two authors, and the
%% "authornote" and "authornotemark" commands
%% used to denote shared contribution to the research.



% \author{Tingliang Zhang}
% \authornote{Both authors contributed equally to this research.}
% \email{ztl20@mails.tsinghua.edu.cn}
% \orcid{0000-0002-0164-4700}
% \author{Hua Tong}
% \authornotemark[1]
% \email{Constanzatong@gmail.com}
% \affiliation{%
%   \institution{Tsinghua University}
%   \streetaddress{Qinghuayuan, Haidian District}
%   \city{Beijing}
%   \country{China}
%   \postcode{100084}
% }

% \author{Yitong Wang}
% \authornote{Both authors contributed equally to this research.}
% \email{3bastet@163.com}
% \author{Meiqi Tu}
% \authornotemark[2]
% \email{@gmail.com}
% \affiliation{%
%   \institution{Tsinghua University}
%   \streetaddress{Qinghuayuan, Haidian District}
%   \city{Beijing}
%   \country{China}
%   \postcode{100084}
% }

% \author{Danni Liu}
% \affiliation{%
%   \institution{University of Washington}
%   \streetaddress{Qinghuayuan, Haidian District}
%   \city{Seattle}
%   \state{Washington}
%   \country{United States}
% }
% \email{dnliudanni@gmail.com}



% \author{Haipeng Mi}
% \affiliation{%
%   \institution{Tsinghua University}
%   \streetaddress{Qinghuayuan, Haidian District}
%   \city{Beijing}
%   \country{China}
%   \postcode{100084}
% }
% \email{mhp@mail.tsinghua.edu.cn}

%%
%% By default, the full list of authors will be used in the page
%% headers. Often, this list is too long, and will overlap
%% other information printed in the page headers. This command allows
%% the author to define a more concise list
%% of authors' names for this purpose.
\renewcommand{\shortauthors}{Zhang and Tong, et al.}

%%
%% The abstract is a short summary of the work to be presented in the
%% article.
\begin{abstract}

  Mental health condition is a major challenge throughout the world, yet mental health services in many countries are struggling to meet such needs. Studies have shown innovative intervention can have positive impacts on patients' mental health conditions. This paper presents SUSI GENE, an egg-shaped portable robot, designed for people with mood disorders, including major depressive disorder, bipolar disorder, etc. Through interactions, SUSI GENE attempts to help patients increase their self-awarenesses, vent their emotions, face their inner conflicts, and reappraise their problems in a less negative approach.

\end{abstract}

  % SUSI GENE is an egg-shaped interactive robot that designed for mental disordered people. It can be placed on the back of  a smart phone.SUSI is a safe, friendly and portable tool to prevent depression disorder and assist patients with mental disorders (such as depression disorder). It help people by increasing self-awareness, assisting treatment and seeking help. It encourages people who are experiencing emotional problems to expose and face their inner vulnerability in an appropriate way, so as to help them vent emotions and carry out cognitive reappraisal.

%%
%% The code below is generated by the tool at http://dl.acm.org/ccs.cfm.
%% Please copy and paste the code instead of the example below.
%%
\begin{CCSXML}
  <ccs2012>
     <concept>
         <concept_id>10003120.10003123.10010860.10010858</concept_id>
         <concept_desc>Human-centered computing~User interface design</concept_desc>
         <concept_significance>500</concept_significance>
         </concept>
     <concept>
         <concept_id>10003456.10010927.10003616</concept_id>
         <concept_desc>Social and professional topics~People with disabilities</concept_desc>
         <concept_significance>500</concept_significance>
         </concept>
     <concept>
         <concept_id>10010583.10010584.10010587</concept_id>
         <concept_desc>Hardware~PCB design and layout</concept_desc>
         <concept_significance>300</concept_significance>
         </concept>
     <concept>
         <concept_id>10003120.10003121.10003125.10010597</concept_id>
         <concept_desc>Human-centered computing~Sound-based input / output</concept_desc>
         <concept_significance>300</concept_significance>
         </concept>
     <concept>
         <concept_id>10010405.10010444.10010446</concept_id>
         <concept_desc>Applied computing~Consumer health</concept_desc>
         <concept_significance>500</concept_significance>
         </concept>
   </ccs2012>
\end{CCSXML}

\ccsdesc[500]{Human-centered computing~User interface design}
\ccsdesc[500]{Social and professional topics~People with disabilities}
\ccsdesc[300]{Hardware~PCB design and layout}
\ccsdesc[300]{Human-centered computing~Sound-based input / output}
\ccsdesc[500]{Applied computing~Consumer health}
%%
%% Keywords. The author(s) should pick words that accurately describe
%% the work being presented. Separate the keywords with commas.
\keywords{datasets, neural networks, gaze detection, text tagging}


%%
%% This command processes the author and affiliation and title
%% information and builds the first part of the formatted document.
\maketitle

\section{Introduction}

-what is susi gene

SUSI GENE is an interactive emotion assistant.  It consists of a tangible egg-shaped robot along with an interface. The robot receives vocal inputs from a user; the mobile phone converts that radio to text for natural language processing; while the interface accordingly generates a virtual creature for the user as well as documents these data.

Past research indicated that people with mood disorders demonstrate overall satisfaction with the usage of mobile technology to increase their mental well-being.\ref{proudfoot2010community}

A large variety of products and research prototypes have made it possible for people to self-monitor their mental conditions, but most of these systems are designed as apps on mobile devices, and thus do not involves tangible interactions.

SUSI GENE also incorporates recording capabilities and requires some operations on mobile devices. However, it has several significant differences. We designed SUSI GENE as an egg-shaped portable robot aims to provide the user a more intuitive experience while sharing his or her stories and feelings. For our current prototype, the user is expected to talk directly to the egg and hold the button that corresponding to his or her current emotion. There are eight emotions, each of them is corresponding to a button with a unique shape. After that, the user needs to place the egg on the back of a mobile device and, through the usage of Near-Field-Communication(NFC) technology, wait for the device pairs with the robot to receive and interpret the piece of audio and vibrates as a feedback signal. During that time, the radio is converted to text for natural language processing(NLP), the text would be split into several keywords. The selected keywords would be analyzed in reference to the HowNet and NTUSD sentiment lexicon, and the “emotion gene” is therefore finalized. After a brief vibration, a creature would hatch out and shown on the screen. The picture of the generated creature and the radio of his or her words will be saved for later usage. The user can not only review those past experiences but also share them to his or her friends, family members, or professional counsellors.

The interactive process imitates the natural hatching of the oviparity animals. Our design assumption is that the process of the young break through its shell is especially inspiring and may bring positive impact on the level of enjoyment.

\section{Background}

Mental health conditon, which causes the most Years lost of Desiabilities(YLD) in the whole world (ref), is influenced by many factors. According to Monroe and Simons' model, these factors can be concluded as diathesis (predisposition/vulnerability) and stress (triggers). The model assumes every individual, no matter of what innate diathesis, has possibilities to develop mental health conditon under certain amount of stress. Thus, the proper react mechanism to the event of stress is the main method to reduce individual's possibility of mental health conditon. Based on the interview of 11 subjects who suffer from mental disorder, we locate two mechanisms: low-recognition of stress-caused emotion changes, and emotion-driven social isolating as the most notable improper ones that may raise the possibilities of mental health conditon and continuely worsen when the mental health condition becomes severe. Recent research and products provide solution by replacing the communicate subject from human to artificial inteligence. However, there is little interactive solution focusing on changing these two mechanisms by guiding the individual to apply new actions to increase diathesis. Therefore, we designed SUSI: an robot with tangible interface to help users shadowing their stress event and related emotion changes through oral expression and generate gamificated communication material to share in real-life relationships.

Today, mood disorder, including depression, has became the worldwide leading cause of the Years Lived with Disability (YLDs). Many countries have started to pay increased attentions to people’s mental health conditions, and a number of plans aimed to make mental health services more accessible have emerged. However, these approaches, including one-to-one counseling, are mostly resource-intensive since each patient should be addressed individually.





\section{Hardware Design}

SUSI Gene is an egg-shaped portable robot, its dimensions are 62 mm in diameter and 80 mm in height. Its center of gravity is so low, that it can stand on the back of the phone like a tumbler.

The SUSI Gene prototype is comprised of three main hardware components: the main PCB with an Arduino NANO BLE Sense and other necessary components on it, a battery, and a 3D printed shell.

SUSI Gene is powered by a 450mAh 2S 7.4V LiPo battery. Most of the power in the robots are consumed by the LEDs and Arduino. The current draw is approximately 200 mA during typical use. Thus, with a 450 mAh battery, SUSI Gene is capable of working for about 2 hours without NFC wireless charging.

SUSI Gene is illuminated in RGBW using WS2812B which are wrapped inside the 3D printed enclosure to provide the robot’s state display as well as full color indicating.

\section{Discussion and future work}

In the future, we will integrate NLP process in SUSI GENE itself by using more powerful chip supporting tenserflow or other machine learning algorithms. 

\end{document}
\endinput
%%
%% End of file `sample-manuscript.tex'.
